\documentclass[14pt,xcolor=pdftex,dvipsnames,table]{beamer}\usepackage[]{graphicx}\usepackage[]{color}
%% maxwidth is the original width if it is less than linewidth
%% otherwise use linewidth (to make sure the graphics do not exceed the margin)
\makeatletter
\def\maxwidth{ %
  \ifdim\Gin@nat@width>\linewidth
    \linewidth
  \else
    \Gin@nat@width
  \fi
}
\makeatother

\definecolor{fgcolor}{rgb}{0.345, 0.345, 0.345}
\newcommand{\hlnum}[1]{\textcolor[rgb]{0.686,0.059,0.569}{#1}}%
\newcommand{\hlstr}[1]{\textcolor[rgb]{0.192,0.494,0.8}{#1}}%
\newcommand{\hlcom}[1]{\textcolor[rgb]{0.678,0.584,0.686}{\textit{#1}}}%
\newcommand{\hlopt}[1]{\textcolor[rgb]{0,0,0}{#1}}%
\newcommand{\hlstd}[1]{\textcolor[rgb]{0.345,0.345,0.345}{#1}}%
\newcommand{\hlkwa}[1]{\textcolor[rgb]{0.161,0.373,0.58}{\textbf{#1}}}%
\newcommand{\hlkwb}[1]{\textcolor[rgb]{0.69,0.353,0.396}{#1}}%
\newcommand{\hlkwc}[1]{\textcolor[rgb]{0.333,0.667,0.333}{#1}}%
\newcommand{\hlkwd}[1]{\textcolor[rgb]{0.737,0.353,0.396}{\textbf{#1}}}%

\usepackage{framed}
\makeatletter
\newenvironment{kframe}{%
 \def\at@end@of@kframe{}%
 \ifinner\ifhmode%
  \def\at@end@of@kframe{\end{minipage}}%
  \begin{minipage}{\columnwidth}%
 \fi\fi%
 \def\FrameCommand##1{\hskip\@totalleftmargin \hskip-\fboxsep
 \colorbox{shadecolor}{##1}\hskip-\fboxsep
     % There is no \\@totalrightmargin, so:
     \hskip-\linewidth \hskip-\@totalleftmargin \hskip\columnwidth}%
 \MakeFramed {\advance\hsize-\width
   \@totalleftmargin\z@ \linewidth\hsize
   \@setminipage}}%
 {\par\unskip\endMakeFramed%
 \at@end@of@kframe}
\makeatother

\definecolor{shadecolor}{rgb}{.97, .97, .97}
\definecolor{messagecolor}{rgb}{0, 0, 0}
\definecolor{warningcolor}{rgb}{1, 0, 1}
\definecolor{errorcolor}{rgb}{1, 0, 0}
\newenvironment{knitrout}{}{} % an empty environment to be redefined in TeX

\usepackage{alltt}

% Specify theme
\usetheme{Madrid}
% See deic.uab.es/~iblanes/beamer_gallery/index_by_theme.html for other themes
\usepackage{caption}
\usepackage{tikz}
 \usetikzlibrary{arrows,positioning}
\usepackage{multirow}
% Specify base color
\usecolortheme[named=OliveGreen]{structure}
% See http://goo.gl/p0Phn for other colors

% Specify other colors and options as required
\setbeamercolor{alerted text}{fg=Maroon}
\setbeamertemplate{items}[square]

% Title and author information
\title{Equity}
\author{Rob Hayward}
\IfFileExists{upquote.sty}{\usepackage{upquote}}{}
\begin{document}

\begin{frame}
\titlepage
\end{frame}

\begin{frame}{Intrinsic value}
Stock investment is easy
\begin{itemize}[<+-| alert@+>]
\pause
\item Find the value of the company
\begin{itemize}
\item If it is under-valued - \textbf{BUY!}
\item If it is over-valued - \textbf{SELL!}
\end{itemize}
\end{itemize}
\end{frame}

\section{Asymmetries of information}
\begin{frame}{Business trends}
Two trends exacerbate asymmetries
\begin{itemize}[<+-| alert@+>]
\pause
\item Rise of large, globalised, public companies
\begin{itemize}
\item Professional managers
\item Complex operations
\end{itemize}
\item Modern fund management
\begin{itemize}
\item Remote owners of the firm
\item Diversified portfolios
\end{itemize}
\end{itemize}
\end{frame}

\begin{frame}{Asymmetries 1}
Two forces that hope to reduce asymmetries
\pause
\begin{itemize}[<+-| alert@+>]
\item Regulatory reporting
\begin{itemize}
\item Quarterly reports of activity
\item Income statement, balance sheet, report on performance
\end{itemize}
\item Professional analysts
\begin{itemize}
\item Use economies of scale and specialisation to understand firms
\item Foster relations with firms
\item Incentives (Investment banks/rating agencies)
\end{itemize}
\end{itemize}
\end{frame}

\begin{frame}{Asymmetries 2}
Problems
\pause
\begin{itemize}[<+-| alert@+>]
\item Accounting irregulatries
\begin{itemize}
\item Enron, Worldcom 
\item Accruals
\begin{itemize}
\item discretionary accruals
\item book revenue now charge costs late
\end{itemize}
\item real manipulations
\begin{itemize}
\item Price cuts
\item R\&D expenditure and investments
\end{itemize}
\end{itemize}
\end{itemize}
\end{frame}

\begin{frame}{Valuing firms}
There are a number of methods that can be used
\begin{itemize}[<+-| alert@+>]
\pause
\item Accounting methods
\item Free-cash flows
\item Relative value and PE
\end{itemize}
\end{frame}

\begin{frame}{Accounting value}
Naive or simplistic
\begin{itemize}[<+-| alert@+>]
\pause
\item What are the value of the assets 
\item Book value
\item Re-sale value of the assets provides some lower limit to valuation
\end{itemize}
\end{frame}

\begin{frame}{Free cash flow}
A more realistic view of firm value is 
\pause
\begin{block}{}
What does the firm do with the assets - how much money does it generate?
\end{block}
\pause
NPV of free cash
\end{frame}

\begin{frame}{Relative value}
\begin{block}{PE ratio}
\begin{equation*}
\frac{P}{E} = \frac{\text{price per share}}{\text{earnings per share}}
\end{equation*}
\end{block}
\end{frame}

\begin{frame}{Historic PE}
\begin{itemize}[<+-| alert@+>]
\item Historic PE ratio of the S\&P 500 is about 14
\begin{itemize} 
\item Anything above this is \emph{overvalued} or \emph{A growth stock}
\item Anything below this is \emph{undervalued} or \emph{A value stock}
\end{itemize}
\item Firms may be highly valued because they are very good
\item Firms may be low valued because they are poor
\end{itemize}
\end{frame}


\section{Equity valuation}
\begin{frame}{Equities and the economy}
There are two broad approaches to valuation
\begin{itemize}[<+-| alert@+>]
\pause
\item Bottom up:  this is the corporate finance rout of understanding the performance of individual companies and amalgamating that to get an overall view of the market
\item Top down: evaluating equities relative to the economy
\end{itemize}
\end{frame}

\begin{frame}{Methods of evaluation}
There are three broad ways of looking at value
\pause
\begin{itemize}[<+-| alert@+>]
\item Cyclically adjusted price earnings ratio (CAPE). 
\item Tobin's Q
\item The equity risk premium
\end{itemize}
\end{frame}

\begin{frame}{CAPE}
Used by Robert Shiller in \emph{Irrational Exuberance}
\pause
\begin{block}{}
\begin{equation*}
CAPE = \frac{\text{S\&P500 real price}}{\text{10-year MA S\&P 500 earnings}}
\end{equation*}
\end{block}
\end{frame}

\begin{frame}{CAPE}
\begin{knitrout}
\definecolor{shadecolor}{rgb}{0.969, 0.969, 0.969}\color{fgcolor}\begin{figure}
\includegraphics[width=\maxwidth]{figure/earnings-1} \caption[Cumulative, adjusted price earnings ratio]{Cumulative, adjusted price earnings ratio\label{fig:earnings}}
\end{figure}


\end{knitrout}
\end{frame}

\begin{frame}{Tobin's Q}
The market value of equity relative to the replacement cost of capital
\begin{block}{}
\begin{equation*}
Q = \frac{\text{Market value}}{\text{Corporate net worth}}
\end{equation*}
\end{block}
\end{frame}

\begin{frame}{Equity risk premium}
The equity risk premium is the return required for taking risk.
\begin{block}{}
\begin{equation}
ERP = R_e - R_{rf} \quad \text{or} \quad R_e - R_b
\end{equation}
\end{block}
Where $ERP$ is the equity risk premium; $R_e$ is the return on equity; $R_fr$ is the risk-free return and $R_b$ is the return on bonds. 
\end{frame}

\begin{frame}{Fed Model}
The Fed model compares the stock market's earnings yield (E/P) to the yield on long-term government bonds. 
\pause
\begin{block}{}
\begin{equation*}
\frac{E}{P} = Y_{10}
\end{equation*}
\end{block}
\pause
Greenspan refers to the link between the fall in the real yield on fixed income that occured since 1990 and the increase in the P/E ratio and return on housing.
\end{frame}

\end{document}
