\documentclass[12pt, a4paper, oneside]{article}\usepackage[]{graphicx}\usepackage[]{color}
%% maxwidth is the original width if it is less than linewidth
%% otherwise use linewidth (to make sure the graphics do not exceed the margin)
\makeatletter
\def\maxwidth{ %
  \ifdim\Gin@nat@width>\linewidth
    \linewidth
  \else
    \Gin@nat@width
  \fi
}
\makeatother

\definecolor{fgcolor}{rgb}{0.345, 0.345, 0.345}
\newcommand{\hlnum}[1]{\textcolor[rgb]{0.686,0.059,0.569}{#1}}%
\newcommand{\hlstr}[1]{\textcolor[rgb]{0.192,0.494,0.8}{#1}}%
\newcommand{\hlcom}[1]{\textcolor[rgb]{0.678,0.584,0.686}{\textit{#1}}}%
\newcommand{\hlopt}[1]{\textcolor[rgb]{0,0,0}{#1}}%
\newcommand{\hlstd}[1]{\textcolor[rgb]{0.345,0.345,0.345}{#1}}%
\newcommand{\hlkwa}[1]{\textcolor[rgb]{0.161,0.373,0.58}{\textbf{#1}}}%
\newcommand{\hlkwb}[1]{\textcolor[rgb]{0.69,0.353,0.396}{#1}}%
\newcommand{\hlkwc}[1]{\textcolor[rgb]{0.333,0.667,0.333}{#1}}%
\newcommand{\hlkwd}[1]{\textcolor[rgb]{0.737,0.353,0.396}{\textbf{#1}}}%

\usepackage{framed}
\makeatletter
\newenvironment{kframe}{%
 \def\at@end@of@kframe{}%
 \ifinner\ifhmode%
  \def\at@end@of@kframe{\end{minipage}}%
  \begin{minipage}{\columnwidth}%
 \fi\fi%
 \def\FrameCommand##1{\hskip\@totalleftmargin \hskip-\fboxsep
 \colorbox{shadecolor}{##1}\hskip-\fboxsep
     % There is no \\@totalrightmargin, so:
     \hskip-\linewidth \hskip-\@totalleftmargin \hskip\columnwidth}%
 \MakeFramed {\advance\hsize-\width
   \@totalleftmargin\z@ \linewidth\hsize
   \@setminipage}}%
 {\par\unskip\endMakeFramed%
 \at@end@of@kframe}
\makeatother

\definecolor{shadecolor}{rgb}{.97, .97, .97}
\definecolor{messagecolor}{rgb}{0, 0, 0}
\definecolor{warningcolor}{rgb}{1, 0, 1}
\definecolor{errorcolor}{rgb}{1, 0, 0}
\newenvironment{knitrout}{}{} % an empty environment to be redefined in TeX

\usepackage{alltt} % Paper size, default font size and one-sided paper
%\graphicspath{{./Figures/}} % Specifies the directory where pictures are stored
%\usepackage[dcucite]{harvard}
\usepackage{rotating}
\usepackage{amsmath}
\usepackage{setspace}
\usepackage{pdflscape}
\usepackage[flushleft]{threeparttable}
\usepackage{multirow}
\usepackage[comma, sort&compress]{natbib}% Use the natbib reference package - read up on this to edit the reference style; if you want text (e.g. Smith et al., 2012) for the in-text references (instead of numbers), remove 'numbers' 
\usepackage{graphicx}
%\bibliographystyle{plainnat}
\bibliographystyle{agsm}
\usepackage[colorlinks = true, citecolor = blue, linkcolor = blue]{hyperref}
%\hypersetup{urlcolor=blue, colorlinks=true} % Colors hyperlinks in blue - change to black if annoying
%\renewcommand[\harvardurl]{URL: \url}
\IfFileExists{upquote.sty}{\usepackage{upquote}}{}
\begin{document}
\title{MCMC}
%\author{Rob Hayward\footnote{University of Brighton Business School, Lewes Road, Brighton, BN2 4AT; Telephone 01273 642586.  rh49@brighton.ac.uk}}
\date{\today}
\maketitle
\section*{Introduction acknowledgement and thanks}
This is a work that is developed from Dave Giles excellent and very helpful overview of Markov Chain Monte Carlo (MCMC) methods. The first of four fantastic posts begins \href{http://davegiles.blogspot.co.uk/2014/03/mcmc-for-econometrics-students-i.html}{here}. 

\section{Bayesian Methods}
There is a blog post from Dave Giles that runs through the Bayesian method.  The example looks at a consumption function (data as consump.dat.txt in the Data folder)
\section{Markov Chain and Gibbs sampler}. Dave Giles code is in the R folder and is called Consumption.R.  \href{http://davegiles.blogspot.ca/2012/04/bayesian-consumption-function.html}{Dave Giles blog post on Bayesian method}. 

\section{Markov Chain}
A Markov chain is a stochasic process where the current value depends only on the immeditely preceeding case. It does not depend on anything before that. The Gibbs sampler.  With two parameters $\Theta_1$ and $\Theta_2$, $p(\Theta_1, \Theta_2)$ is the prior pdf and $L(\Theta_1, \Theta_2 | y) = p(y | \Theta_1, \Theta_2)$ is the likelihood function. By Bayes Theory, the posterior pdf is 
\begin{equation}
p(\Theta_1, \Theta_2 | y) \propto p(\Theta_1, \Theta_2)L(\Theta_1, \Theta_2 |y)
\end{equation}


\end{document}
