\documentclass[14pt,xcolor=pdftex,dvipsnames,table]{beamer}\usepackage[]{graphicx}\usepackage[]{color}
%% maxwidth is the original width if it is less than linewidth
%% otherwise use linewidth (to make sure the graphics do not exceed the margin)
\makeatletter
\def\maxwidth{ %
  \ifdim\Gin@nat@width>\linewidth
    \linewidth
  \else
    \Gin@nat@width
  \fi
}
\makeatother

\definecolor{fgcolor}{rgb}{0.345, 0.345, 0.345}
\newcommand{\hlnum}[1]{\textcolor[rgb]{0.686,0.059,0.569}{#1}}%
\newcommand{\hlstr}[1]{\textcolor[rgb]{0.192,0.494,0.8}{#1}}%
\newcommand{\hlcom}[1]{\textcolor[rgb]{0.678,0.584,0.686}{\textit{#1}}}%
\newcommand{\hlopt}[1]{\textcolor[rgb]{0,0,0}{#1}}%
\newcommand{\hlstd}[1]{\textcolor[rgb]{0.345,0.345,0.345}{#1}}%
\newcommand{\hlkwa}[1]{\textcolor[rgb]{0.161,0.373,0.58}{\textbf{#1}}}%
\newcommand{\hlkwb}[1]{\textcolor[rgb]{0.69,0.353,0.396}{#1}}%
\newcommand{\hlkwc}[1]{\textcolor[rgb]{0.333,0.667,0.333}{#1}}%
\newcommand{\hlkwd}[1]{\textcolor[rgb]{0.737,0.353,0.396}{\textbf{#1}}}%

\usepackage{framed}
\makeatletter
\newenvironment{kframe}{%
 \def\at@end@of@kframe{}%
 \ifinner\ifhmode%
  \def\at@end@of@kframe{\end{minipage}}%
  \begin{minipage}{\columnwidth}%
 \fi\fi%
 \def\FrameCommand##1{\hskip\@totalleftmargin \hskip-\fboxsep
 \colorbox{shadecolor}{##1}\hskip-\fboxsep
     % There is no \\@totalrightmargin, so:
     \hskip-\linewidth \hskip-\@totalleftmargin \hskip\columnwidth}%
 \MakeFramed {\advance\hsize-\width
   \@totalleftmargin\z@ \linewidth\hsize
   \@setminipage}}%
 {\par\unskip\endMakeFramed%
 \at@end@of@kframe}
\makeatother

\definecolor{shadecolor}{rgb}{.97, .97, .97}
\definecolor{messagecolor}{rgb}{0, 0, 0}
\definecolor{warningcolor}{rgb}{1, 0, 1}
\definecolor{errorcolor}{rgb}{1, 0, 0}
\newenvironment{knitrout}{}{} % an empty environment to be redefined in TeX

\usepackage{alltt}

% Specify theme
\usetheme{Madrid}
% See deic.uab.es/~iblanes/beamer_gallery/index_by_theme.html for other themes
\usepackage{caption}
\usepackage[comma, sort&compress]{natbib}
\usepackage{graphicx}
\usepackage{amsmath}
\bibliographystyle{agsm}
% Specify base color
\usecolortheme[named=OliveGreen]{structure}
% See http://goo.gl/p0Phn for other colors

% Specify other colors and options as required
\setbeamercolor{alerted text}{fg=Maroon}
\setbeamertemplate{items}[square]

% Title and author information
\title{Pairs Trading - Relative Value}
\author{Rob Hayward}
\IfFileExists{upquote.sty}{\usepackage{upquote}}{}


\begin{document}

\begin{frame}
\titlepage
\end{frame}

\begin{frame}{Outline}
\tableofcontents
\end{frame}

\section{Introduction}
\begin{frame}{Introduction}
This is a return full circle to the original hedge funds
\begin{itemize}[<+-| alert@+>]
\item Pairs trading will hope to provide an \emph{absolute return}
\item Paris trading can remove market risk and leave exposure to specific risk
\item Depends upon being able to short securities
\begin{itemize}
\item Professionl ability to borrow securities
\item Use futures or contract-for-difference
\end{itemize}
\end{itemize}
\end{frame}

\section{Pairs Trading}
\begin{frame}{Pairs Trading}
The trade depends on a temporary brekdown in the correlation between two similar securities
\begin{itemize}[<+-| alert@+>]
\item LTCM strategy
\item When the usual relationship breaks down, speculate on a return to normality
\item Exmples
\begin{itemize}
\item Tesco and Sainsbury
\item Coke and Pepsie
\item 5-year bond and 10-year bond
\item Gold and silver
\end{itemize}
\end{itemize}
\end{frame}

%\begin{frame}{Pairs}
%<<Coke, echo=FALSE>>=
%da <- read.csv('http://www.quandl.com/api/v1/datasets/GOOG/NYSE_KO.csv?&auth_token=mUCjthkJFQDsYVrFh4Gh&trim_start=2000-01-01&trim_end=2013-12-31&sort_order=desc', colClasses=c('Date'='Date'))
%da1 <- ead.csv('http://www.quandl.com/api/v1/datasets/GOOG/NYSE_PEP.csv?&auth_token=mUCjthkJFQDsYVrFh4Gh&trim_start=2000-01-01&trim_end=2013-12-31&sort_order=desc', colClasses=c('Date'='Date'))
%head(da)
%@
%\end{frame}

\begin{frame}{Cointegration}
One method to look at the relationship would be the assess whether the two are cointegrated.
\begin{itemize}[<+-| alert@+>]
\item Can use the Engle-Granger method and check that the residuals from the regression are \emph{stationary}
\item If the residuals are stationary, we expect the previous relationship to be restored
\item An \emph{Error-Correction Model} can be used to assess the speed of the return to the set relationship
\end{itemize}
\end{frame}




\section{Relative Value}
\begin{frame}{Relative value}
Now the focus is on one part of the pair out-performing
This can be based on, for example
\begin{itemize}[<+-| alert@+>]
\item Reltive PE ratios
\item An assessment of competitive advantage
\item Products and brands
\item Country risk
\item Yield curve, durtion and changes in short-term interest rates
\end{itemize}

\end{frame}



\end{document}
