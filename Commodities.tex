\documentclass[12pt, a4paper, oneside]{article}\usepackage[]{graphicx}\usepackage[]{color}
%% maxwidth is the original width if it is less than linewidth
%% otherwise use linewidth (to make sure the graphics do not exceed the margin)
\makeatletter
\def\maxwidth{ %
  \ifdim\Gin@nat@width>\linewidth
    \linewidth
  \else
    \Gin@nat@width
  \fi
}
\makeatother

\definecolor{fgcolor}{rgb}{0.345, 0.345, 0.345}
\newcommand{\hlnum}[1]{\textcolor[rgb]{0.686,0.059,0.569}{#1}}%
\newcommand{\hlstr}[1]{\textcolor[rgb]{0.192,0.494,0.8}{#1}}%
\newcommand{\hlcom}[1]{\textcolor[rgb]{0.678,0.584,0.686}{\textit{#1}}}%
\newcommand{\hlopt}[1]{\textcolor[rgb]{0,0,0}{#1}}%
\newcommand{\hlstd}[1]{\textcolor[rgb]{0.345,0.345,0.345}{#1}}%
\newcommand{\hlkwa}[1]{\textcolor[rgb]{0.161,0.373,0.58}{\textbf{#1}}}%
\newcommand{\hlkwb}[1]{\textcolor[rgb]{0.69,0.353,0.396}{#1}}%
\newcommand{\hlkwc}[1]{\textcolor[rgb]{0.333,0.667,0.333}{#1}}%
\newcommand{\hlkwd}[1]{\textcolor[rgb]{0.737,0.353,0.396}{\textbf{#1}}}%

\usepackage{framed}
\makeatletter
\newenvironment{kframe}{%
 \def\at@end@of@kframe{}%
 \ifinner\ifhmode%
  \def\at@end@of@kframe{\end{minipage}}%
  \begin{minipage}{\columnwidth}%
 \fi\fi%
 \def\FrameCommand##1{\hskip\@totalleftmargin \hskip-\fboxsep
 \colorbox{shadecolor}{##1}\hskip-\fboxsep
     % There is no \\@totalrightmargin, so:
     \hskip-\linewidth \hskip-\@totalleftmargin \hskip\columnwidth}%
 \MakeFramed {\advance\hsize-\width
   \@totalleftmargin\z@ \linewidth\hsize
   \@setminipage}}%
 {\par\unskip\endMakeFramed%
 \at@end@of@kframe}
\makeatother

\definecolor{shadecolor}{rgb}{.97, .97, .97}
\definecolor{messagecolor}{rgb}{0, 0, 0}
\definecolor{warningcolor}{rgb}{1, 0, 1}
\definecolor{errorcolor}{rgb}{1, 0, 0}
\newenvironment{knitrout}{}{} % an empty environment to be redefined in TeX

\usepackage{alltt} % Paper size, default font size and one-sided paper
%\graphicspath{{./Figures/}} % Specifies the directory where pictures are stored
%\usepackage[dcucite]{harvard}
\usepackage{rotating}
\usepackage{amsmath}
\usepackage{setspace}
\usepackage{pdflscape}
\usepackage[flushleft]{threeparttable}
\usepackage{multirow}
\usepackage[comma, sort&compress]{natbib}% Use the natbib reference package - read up on this to edit the reference style; if you want text (e.g. Smith et al., 2012) for the in-text references (instead of numbers), remove 'numbers' 
\usepackage{graphicx}
%\bibliographystyle{plainnat}
\bibliographystyle{agsm}
\usepackage[colorlinks = true, citecolor = blue, linkcolor = blue]{hyperref}
%\hypersetup{urlcolor=blue, colorlinks=true} % Colors hyperlinks in blue - change to black if annoying
%\renewcommand[\harvardurl]{URL: \url}
\IfFileExists{upquote.sty}{\usepackage{upquote}}{}
\begin{document}
\title{Commodity Futures}
%\author{Rob Hayward\footnote{University of Brighton Business School, Lewes Road, Brighton, BN2 4AT; Telephone 01273 642586.  rh49@brighton.ac.uk}}
\date{\today}
\maketitle
\section*{Introduction}
Commodities are primarily traded in futures markets.  These are an agreement now to pay a particular price for a particular commodity at some set date in the future. These markets are mostly exchange traded and are therefore the quantities to be supplied and the dates at which this will be done are standardised.  Standardisation makes the contract more widely applicable and increases liquidity.  The difference between the underlying risk that is to be hedge by these contracts and the futures is called the \emph{basis risk}.  

The futures market is divided into two types of activity:  hedging and speculation.  The hedgers aim to reduce risk by locking in the price that they will buy or sell the commodity in the future.  For example, the oil hedgers will be oil companies, airline companies and other producers and users of oil and related products. The speculators taking a position in the market because they believe that they can make a profit from price movement.  Keynes and Hicks argued that futures prices would tend to trade at a discount to the spot price because speculators would require some payment for taking risk from the hedgers. When the futures price is above the spot price or the price of the near future is above the far contract, this is called \emph{contango}; when the spot price is above the future or the near future is above the far future, this is called \emph{backwardation}

The US Commodity and Futures Trading Commission (CFTC), a regulatory body, requires that all participants in the futures market categorise themseleves as \emph{commerical} with some underlying business interest in the 

\section*{Pricing futures contracts}
The price of a futures contract should be equal to the cost of buying the commodity now and storing it until the delivery date.  For a financial serucity where the only cost is the cost of finance, the $F(t, T)$ futures price for time $T$, valued at $t$, where $t<T$ is equal to, 
\begin{equation*}
F(t, T) = S(t) \times (1 +r)^{(T-t)}
\end{equation*}
or, in continuous time, 
\begin{equation*}
F(t, T) = S(t)e^{r(T-t)}
\end{equation*}
For other futures, storage costs, income in the form of dividends and coupons and any other benefits that accrue from the holding of the commodity. This may be particularly important if there are large storage costs involved for commodities.

When there are disruptions in the supply of the commodity, this arbitrage mechanism will not work. In that case, the pricing will be determined by expections. 
\begin{equation*}
F(t, T) = E_t[S(T)]
\end{equation*}





The evolution of specultive positions shows that these positions have risen to near record levels.   
\begin{knitrout}
\definecolor{shadecolor}{rgb}{0.969, 0.969, 0.969}\color{fgcolor}

{\centering \includegraphics[width=\maxwidth]{figure/Oilspec} 

}



\end{knitrout}

However, the oil price remains below the highest levels. 

\begin{knitrout}
\definecolor{shadecolor}{rgb}{0.969, 0.969, 0.969}\color{fgcolor}
\includegraphics[width=\maxwidth]{figure/oilprice} 

\end{knitrout}


\section*{Delivery}
The futures contracts will expire in set months of March, June, September and December.  The actual time is determined by the exchange. 

Supply and Demand

Trade the spread




\end{document}
