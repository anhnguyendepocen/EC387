\documentclass[14pt,xcolor=pdftex,dvipsnames,table]{beamer}\usepackage[]{graphicx}\usepackage[]{color}
%% maxwidth is the original width if it is less than linewidth
%% otherwise use linewidth (to make sure the graphics do not exceed the margin)
\makeatletter
\def\maxwidth{ %
  \ifdim\Gin@nat@width>\linewidth
    \linewidth
  \else
    \Gin@nat@width
  \fi
}
\makeatother

\definecolor{fgcolor}{rgb}{0.345, 0.345, 0.345}
\newcommand{\hlnum}[1]{\textcolor[rgb]{0.686,0.059,0.569}{#1}}%
\newcommand{\hlstr}[1]{\textcolor[rgb]{0.192,0.494,0.8}{#1}}%
\newcommand{\hlcom}[1]{\textcolor[rgb]{0.678,0.584,0.686}{\textit{#1}}}%
\newcommand{\hlopt}[1]{\textcolor[rgb]{0,0,0}{#1}}%
\newcommand{\hlstd}[1]{\textcolor[rgb]{0.345,0.345,0.345}{#1}}%
\newcommand{\hlkwa}[1]{\textcolor[rgb]{0.161,0.373,0.58}{\textbf{#1}}}%
\newcommand{\hlkwb}[1]{\textcolor[rgb]{0.69,0.353,0.396}{#1}}%
\newcommand{\hlkwc}[1]{\textcolor[rgb]{0.333,0.667,0.333}{#1}}%
\newcommand{\hlkwd}[1]{\textcolor[rgb]{0.737,0.353,0.396}{\textbf{#1}}}%

\usepackage{framed}
\makeatletter
\newenvironment{kframe}{%
 \def\at@end@of@kframe{}%
 \ifinner\ifhmode%
  \def\at@end@of@kframe{\end{minipage}}%
  \begin{minipage}{\columnwidth}%
 \fi\fi%
 \def\FrameCommand##1{\hskip\@totalleftmargin \hskip-\fboxsep
 \colorbox{shadecolor}{##1}\hskip-\fboxsep
     % There is no \\@totalrightmargin, so:
     \hskip-\linewidth \hskip-\@totalleftmargin \hskip\columnwidth}%
 \MakeFramed {\advance\hsize-\width
   \@totalleftmargin\z@ \linewidth\hsize
   \@setminipage}}%
 {\par\unskip\endMakeFramed%
 \at@end@of@kframe}
\makeatother

\definecolor{shadecolor}{rgb}{.97, .97, .97}
\definecolor{messagecolor}{rgb}{0, 0, 0}
\definecolor{warningcolor}{rgb}{1, 0, 1}
\definecolor{errorcolor}{rgb}{1, 0, 0}
\newenvironment{knitrout}{}{} % an empty environment to be redefined in TeX

\usepackage{alltt}

% Specify theme
\usetheme{Madrid}
% See deic.uab.es/~iblanes/beamer_gallery/index_by_theme.html for other themes
\usepackage{caption}
\usepackage{tikz}
 \usetikzlibrary{arrows,positioning}
\usepackage{multirow}
% Specify base color
\usecolortheme[named=OliveGreen]{structure}
% See http://goo.gl/p0Phn for other colors

% Specify other colors and options as required
\setbeamercolor{alerted text}{fg=Maroon}
\setbeamertemplate{items}[square]

% Title and author information
\title{Futures, forwards and carry}
\author{Rob Hayward}
\IfFileExists{upquote.sty}{\usepackage{upquote}}{}
\begin{document}

\begin{frame}
\titlepage
\end{frame}

\section{Future markets}
\begin{frame}{Futures}
A market to agree a price now for future delivery
\begin{itemize}[<+-| alert@+>]
\item Keynes and Hicks argue that spot should trade at a discount to future so that there is a profit for speculators.
\item This is a return for taking risk and providing liquidity
\item Future $>$ spot is \emph{contango}
\item Future $<$ spot is \emph{backwardation} 
\end{itemize}
\end{frame}

\begin{frame}{Price of futures 1}
When the only cost is the cost of finance
\begin{block}{}
\begin{equation}
F(t, T) = S(t) \times (1+r)^{(T-t)}
\end{equation}
\end{block}
or, in continuous time
\begin{block}{}
\begin{equation}
F(t, T) = S(t)e^{r(T-t)}
\end{equation}
\end{block}
Where $t$ is today, $T$ is the exercise date, $r$ is the interest rate, $F$ is the future price and $S$ is the spot price. 
\end{frame}

\begin{frame}{Pricing of futures 2}
However, in many cases there are other costs and benefits that have to be included
\begin{itemize}[<+-| alert@+>]
\item Benefits
\begin{itemize}
\item Dividends
\item Coupons
\item Use of commodity as collateral
\end{itemize}
\item Costs
\begin{itemize}
\item Storage costs
\item Insurance
\item Shipping 
\end{itemize}
\end{itemize}
\end{frame}

\section{Uncovered interest parity}
\begin{frame}{Covered interest parity}
\begin{block}{}
\begin{equation*}\label{eqref:uip}
\frac{F_{t, j}}{S_t} \times (1 + i_{t,j}) = (1 + i_{t,j}^*)  
\end{equation*}
\end{block}
Where, $F_{t, j}$ is the forward rate in terms of foreign currency for domestic at time $t$ for $j$ periods ahead, $S_t$ is the spot rate, $i_{t,j}$ is the nominal interest rate at time $t$ for $j$ periods ahead and $i^*$ is the foreign rate.
\end{frame}

\begin{frame}{Forward rate 1}
\begin{block}{}
\begin{equation}\label{eqref:cip}
{F_{t, j}} = {S_t} \times \frac{(1 + i_{t,j}^*)}{(1 + i_{t,j})}  
\end{equation}
\end{block}
Calculate the 1-year forward for USD-JPY with spot at JPY 150; the US 1-year rate at $1.0\%$ and the Japanese 1-year rate at zero. 
\end{frame}

\begin{frame}{Forward rate 2}
\begin{table}[h!]
\begin{center}
\begin{tabular}{l r r}
\textbf{Action} & \textbf{USD} & \textbf{JPY}\\
\hline
Borrow USD & 1,000,000 & \\
Owe & 1,010,000&  \\
Buy JPY & & 150,000,000\\
Receive & & 150,000,000\\
Sell & $\frac{150,000,000}{148.5149} = 1,010,000$ &
\end{tabular}
\end{center}
\caption{Covered interest parity}
\label{tabref:cip1}
\end{table}
\end{frame}

\begin{frame}{Covered interest parity 3}
From Equation \ref{eqref:cip}
\begin{block}{}
\begin{equation*}
\frac{F_{t,j}}{S_t} = \frac{(1 + i_{t, j}^*)}{(1 + i_{t, j})} 
\end{equation*}
\end{block}
Taking 1 from each side and re-arranging.
\begin{block}{}
\begin{equation}\label{eqref:cip3}
\frac{F_{t,j} - S_t}{S_t} = \frac{(i_{t,j}^* - i_{t,j})}{(1 + i_{t,j})} 
\end{equation}
\end{block}
\end{frame}

%\begin{frame}{Covered interest parity 4}
%If the domestic rate of interest is low and the log difference is substituted for the percentage change, Equation \ref{eqref:cip3} can be approximated as 
%\begin{block}{}
%\begin{equation}
%\Delta s_t = i_{t,j}^* - i_{t,j} 
%\end{equation} 
%\end{block}
%Where $\Delta s_t$ is $log (s_t) - log (s_{t-1})$ 
%\end{frame}

\begin{frame}{Uncovered interest parity 1}
If the forward exposure is not \emph{covered} there is a risk of a loss or a gain.
\pause
\begin{itemize}[<+-| alert@+>]
\item This means that the value of the activity will be based on expectations about the future
\item There is a whole literature on \emph{expectations}
\item \emph{Rational expectations} assumes that economic agents form their expectations on the basis of the underlying economic model that is being used
\item In this case, it would assume that investors assume that forward rate will give an \emph{unbiased prediction} of  the future rate
\item If not, there is an arbitrage opportunity
\end{itemize}
\end{frame}

\begin{frame}{Expectations and forward rate 1}
\begin{table}[h!]
\begin{center}
\begin{tabular}{l r r}
\textbf{Action} & \textbf{USD} & \textbf{JPY}\\
\hline
Borrow USD & 1,000,000 & \\
Owe & 1,010,000&  \\
Buy JPY & & 150,000,000\\
Receive & & 150,000,000\\
Sell & $\frac{150,000,000}{150.00} = 1,000,000$ &
\end{tabular}
\end{center}
\caption{Expected rate is above forward: Loss}
\label{tabref:cip2}
\end{table}
\end{frame}

\begin{frame}{Expectations and forward rate 2}
\begin{table}[h!]
\begin{center}
\begin{tabular}{l r r}
\textbf{Action} & \textbf{USD} & \textbf{JPY}\\
\hline
Borrow USD & 1,000,000 & \\
Owe & 1,010,000&  \\
Buy JPY & & 150,000,000\\
Receive & & 150,000,000\\
Sell & $\frac{150,000,000}{145.00} = 1,034,483$ &
\end{tabular}
\end{center}
\caption{Expected rate is below forward: gain}
\label{tabref:cip3}
\end{table}
\end{frame}

\begin{frame}{Uncovered interest parity 2}
The expected future rate is equal to the forward rate so,
\begin{block}{}
\begin{equation}
E[s_{t+j}] - s_t = \frac{(i_{t, j}^* - i_{t, j})}{(1 + i_{t, j})}
\end{equation}
\end{block}
or approximately,
\begin{block}{}
\begin{equation}
E[\Delta s_{t+j}] = i_{t, j}^* - i_{t, j}
\end{equation}
\end{block}
\begin{block}{}
\begin{equation}
\Delta s_t = i_{t,j}^* - i_{t,j} 
\end{equation} 
\end{block}
Where $\Delta s_t$ is $log (s_t) - log (s_{t-1})$ 
\end{frame}

\begin{frame}{Testing UIP 1}
A usual test of UIP is 
\begin{block}{}
\begin{equation}
\label{eq:uip}
\Delta s_{t + j} = \beta_0 +\beta_1 f_{t+j} + \varepsilon
\end{equation} 
\end{block}
Where $f_{t+j}$ is the \emph{forward premium} and UIP would mean that $\beta_0 = 0$ and $\beta_1 = 1$
\begin{block}{}
Many tests find that $\beta_1 < 0$
\end{block}
\end{frame}

\begin{frame}{Testing UIP 2}
UIP does not usually seem to hold
\pause
\begin{itemize}[<+-| alert@+>]
\item Low interest currency does not appreciate vs the high interest rate currency
\item It appears to be beneficial to borrow low interest rate currency and deposit in higher yielding currency
\item \emph{Carry trade}
\item See CHF
\end{itemize}
\end{frame}

\begin{frame}{Testing UIP 3}
\begin{block}{}
\emph{The joint hypothesis problem}: If the test fails, it is not clear whether if is UIP that has failed or our assumption about expectations
\end{block}
\pause
\begin{itemize}[<+-| alert@+>]
\item UIP failure: are there market frictions that prevent arbitrage?   
\item Expectations failure: is there a risk premium that explains the divergence? 
\end{itemize}
\end{frame} 



\begin{frame}{Carry returns in calm and crisis}
\includegraphics[scale = 0.42]{"../../UIPdoc/hist1a"}
\end{frame}

%\begin{frame}{Carry returns in calm and crisis 2}
%\includegraphics[scale = 0.42]{"../../../UIPdoc/hist2a"}
%\end{frame}


\end{document}
