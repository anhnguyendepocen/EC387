\documentclass[14pt,xcolor=pdftex,dvipsnames,table]{beamer}

% Specify theme
\usetheme{Madrid}
% See deic.uab.es/~iblanes/beamer_gallery/index_by_theme.html for other themes
\usepackage{caption}
\usepackage{listing}
% Specify base color
\usecolortheme[named=OliveGreen]{structure}
% See http://goo.gl/p0Phn for other colors

% Specify other colors and options as required
\setbeamercolor{alerted text}{fg=Maroon}
\setbeamertemplate{items}[square]

% Title and author information
\title{Simple and continuously compounded returns}
\author{Rob Hayward}


\begin{document}

\begin{frame}
\titlepage
\end{frame}

\begin{frame}{Outline}
\tableofcontents
\end{frame}


\section{Time Value of Money}
\begin{frame}{Time Value of Money}
$\$FV_n = \$V(1+R)^n$, \\
\vskip1cm
where $\$FV_n$ is the future value after $n$ years 
\end{frame}

\begin{frame}{Example}
£100 invested at 5\% for a number of years
\pause
\begin{itemize}[<+-| alert@+>]
\item $FV_1 = 100(1.05)^1 = 105.00$
\item $FV_5 = 100(1.05)^5 = 127.63$
\item $FV_9 = 100(1.05)^9 = 155.13$
\end{itemize}
\end{frame}

\begin{frame}{Re-arrange}
FV is a function of $FV_n$, $V$, $R$, $n$.   Re-arrange to find
\pause
\begin{block}{Present Value}
\begin{equation}\label{eqref:pv}
V=\frac{FV_n}{(1+R)^n}
\end{equation}
\end{block}
\pause
\begin{block}{Compound return}
\begin{equation}\label{eqref:cr}
R=\left( \frac{FV_n}{V} \right)^{1/n} - 1
\end{equation}
\end{block}
\pause
\begin{block}{Investment horizon}
\begin{equation}\label{eqref:ih}
n=\frac{ln(FV_n/V)}{ln(1+R)}
\end{equation}
\end{block}
\end{frame} 

\begin{frame}{Multiple Compounding}
When interest is paid frequency, 
\begin{equation}
FV^m_n = \$V \left( 1+\frac{R}{m}\right)^{mn}
\end{equation}
where $\frac{R}{m}$ is periodic interest
\begin{equation}\label{eqref:cc}
FV^{\infty}_n=\underset{m \rightarrow \infty}{lim}\$V \left(1+\frac{R}{m} \right)^{mn} = \$Ve^{Rn}
\end{equation}
where $e^1=2.71828$
\end{frame}

\begin{frame}{Example}
With a simple interest rate of 10\%, what is the value at the end of one year (n = 1) for different rates of compounding (values of m)  
\begin{center}
\rowcolors{1}{OliveGreen!20}{OliveGreen!5}
 \begin{tabular}{l | r}
Compounding Frequency & Value at year end (R = 10\%)\\
\hline
Annual (m = 1) & 110.00\\
Semi-annual (m=2) & 110.25\\
Quarterly (m=4) & 110.38\\
Monthly (m=12) & 110.47\\
Daily (m=365) & 110.52\\
Continuous (m= $\infty$) & 110.52\\
\hline
\end{tabular}
\end{center}
\end{frame} 

% Must put [fragile] to use verbatim
\begin{frame}[fragile]{R code}
\begin{verbatim}
Return <- function(R, m, n){
  FV = (1 + R/m)^(m*n)
}
a <- Return(R = 0.1, m = 100000, n =1)

b <- exp(0.1)

Does a = b?
\end{verbatim}
\end{frame}

\begin{frame}{Effective Annual Rate}
Annual Rate $R_A$ that equates $FV_n^m$ with $FV_n$
\begin{equation}
\$V \left(1 + \frac{R}{m} \right)^{mn} = \$V(1+R_A)^n
\end{equation}
Solve for $R_A$
\begin{align}
\left(1+\frac{R}{m}\right)^m &= 1 + R_A \\
R_A &= \left(1+\frac{R}{m}\right)^m -1
\end{align}
\end{frame}

\begin{frame}{Effective annual rate example}
For \$100 with a 10\% return paid quarterly, 
\begin{align}
FV & = V \left (1 + \frac{0.1}{4} \right )^{4\times 1}\\
FV & = V (1 + R_A)\\ 
110.3813 & = 100(1 + R_A)\\
R_A & = \left (\frac{110.3813}{100} \right) -1\\ 
R_A & = 10.3813
\end{align}
\end{frame}

\begin{frame}{Continuous Compounding}
The effective annual rate is greater than the simple rate because of the compounding.  This is seen clearly with continuous compounding
\begin{align}
\$Ve^{Rn} &=\$V(1+R_A)^n\\
 e^R & =(1+R_A)\\
 e^R - 1 & = R_A\\
 R & = ln(1 + R_A)\label{eqref:ccln}
\end{align}
\end{frame}

\begin{frame}{Example}
Annual rates ($R_A$) when there are different levels of compounding (m).  Rate (R) = 10\%.  
\begin{center}
\rowcolors{1}{OliveGreen!20}{OliveGreen!5}
 \begin{tabular}{l | r |r}
Compounding Frequency & Value  & $R_A$\\
\hline
Annual (m = 1) & 110.00 & 10.00\%\\	
Semi-annual (m=2) & 110.25 & 10.25\%\\
Quarterly (m=4) & 110.38 & 10.38\%\\
Monthly (m=12) & 110.47 & 10.47\%\\
Daily (m=365) & 110.52 & 10.52\%\\
Continuous (m= $\infty$) & 110.52 & 10.52\%\\
\hline
\end{tabular}
\end{center}
\end{frame} 



\section{Asset Returns}
\begin{frame}{Simple returns}
\begin{itemize}
\item $P_t$ price at the end of month t
\item $P_{t-1}$ price at the end of month t-1
\end{itemize}
\begin{equation}
R_t=\frac{P_t - P_{t-1}}{P_{t-1}}=\% \Delta P_t
\end{equation}
net monthly return in month t

\begin{equation}
1+R_t=\frac{P_t}{P_{t-1}}
\end{equation}
Gross monthly return in month t
\end{frame}

\begin{frame}{Multi period Returns}
A simple two month return is calculated as 
\begin{align*}
R_t(2) & =\frac{P_{t+2}-P_t}{P_t}\\
&= \frac{P_{t+2}}{P_t} -1\\
& = \frac{P_{t+2}}{P_{t+1}}\frac{P_{t+1}}{P_t} -1\\
&= (1+R_t)(1+R_{t+1})-1
\end{align*}
\end{frame}

\begin{frame}{Multi-period Simple Returns}
If $1+R_t$ is the gross return over month $t$ and $1+R_{t+1}$ is the gross return over month $t+1$, 
\begin{equation}
1+R_t(2) = (1+R_{t+1})(1+R_t)
\end{equation}
Note that the two month gross return is equal to the product of the two one-month gross returns.  It is not equal to the sum of the two one-month returns. 
\begin{align*}
R_t(2) &= R_{t+2}+R_{t+1}+R_{t+2}R_{t+1}\\
& \approx R_{t+2}+R_{t+1}
\end{align*}
if $R_{t+2}$ and $R_{t+1}$ are small 
\end{frame}

\begin{frame}{Example}
Vodafone share price is £22.5 in January, £25.6 in February and £26.76 in March. The two-month return is 
\begin{equation}
R_t(2) = \frac{26.76-22.5}{22.5} = 18.93\%
\end{equation}
\begin{equation}
R_t=\frac{25.6-22.5}{22.5} = 13.78\%
\end{equation}
\begin{equation}
R_{t+1}=\frac{26.76-25.6}{25.6} = 4.53\%
\end{equation}
$1.1893 =1.1378 \times 1.0453$\\
$18.93  \approx 13.87 + 4.53$
\end{frame}

\begin{frame}{Annualising Returns}
Returns are usually converted to an annual rate to ease comparison. \\
Example, if the simple monthly return is 5\%, it is assumed that this is repeated for the next 12 months. 
Compound annual gross return \\
\begin{itemize}
\item $(1+R_A) = 1+R_t(12) = (1+R_m)^{12}$\\
\item Compound net annual return \\
\item $R_A = (1+R_m)^{12} - 1$
\end{itemize}
\end{frame}

\begin{frame}{Example}
Simple monthly return on Microsoft is 2.5\%
\begin{align*}
(1+R_A) &= (1 + R_t(12)) = (1+ R_m)^{12}\\
&= 1.025^{12}\\
&= 1.3449\\
R_A &= 34.49\%
\end{align*}
\end{frame}

\begin{frame}{Total Returns}
Dividend payments between months $t$ and $t+1$
\begin{align*}
R_t^{total} &= \frac{P_{t+1} + D_t - P_t}{P_t} \\ 
& = \frac{P_{t+1}-P_t}{P_t} + \frac{D_t}{P_t}\\
& \text{Capital gain + Dividend yield}
\end{align*}
\end{frame}

\begin{frame}{Example}
Santander share price rises from £35.30 to 35.76 in the month of February and there is a 76 pence per share dividend paid.  
\begin{align*}
R_t^{total} &= \frac{35.76+ 0.76 -35.3}{35.3} \\
& = \frac{35.76-35.3}{35.3} + \frac{0.76}{35.3}\\
& = 3.5\%
\end{align*}
This is made up of 1.3\% capital gain and 2.1\% dividend yield. 
\end{frame}

\begin{frame}{Annualising Two Year Return}
Vodafone stock rises to £34.5 at the close of 2012 from £34.00 at the beginning of the year and £32.78 at the beginning of 2011.  There are no dividends. 
\begin{align*}
1+R_t(24) &= \frac{34.5}{32.78} = 1.0524\\
R_t(24) &= 5.24\%
\end{align*}
If the two year return is 5.24\%, the compound annual  return is 
\begin{align*}
(1 + R_A)^2 &= 1 +R_t(24) = 1.0524\\
R_A &= (1.0524)^{0.5} - 1\\
&= 2.59\%
\end{align*}
\end{frame}


\section{Portfolio Performance}
\begin{frame}{Portfolio Returns}
\begin{itemize}
\item If there is an investment of $£V$ and two securities A and B
\item $x_A$ is the proportion invested in security A
\item $x_B$ is the proportion invested in security B
\item $x_A + x_B = 1$
\end{itemize}
\end{frame}

\begin{frame}{Return on Investment}
\begin{align*}
£V(1+R_{p,t}) &= £V[x_A (1+R_{A,t}) + x_B (1+R_{B, t})]\\
& = £V[x_A + x_B + x_A R_{A,t} + x_B R_{B,t}]\\
& = £V[1 + x_A R_{A,t} + x_B R_{B,t}]\\
\Rightarrow R_{p,t} & = x_A R_{A,t} + x_B R_{B,t}
\end{align*}
\end{frame}

\begin{frame}{Example}
If the fund buys 10 shares in Google at \$345 and 5 shares in Apple at \$678
\begin{itemize}
\item $P_{G,t} = 10 \times 345 = 3450$
\item $P_{A,t} = 5 \times 678 = 3390$
\item Total value of the portfolio $\$3450 + \$3390 = \$6840$
\item $x_{G,t} = \frac{3450}{6840} = 50.44\%$
\item $x_{A,t} = \frac{3390}{6840} = 49.56\%$
\end{itemize}
\end{frame}

\begin{frame}{Example cont.}
If the price of Google rises to \$356 and the price of Apple falls to \$670, the simple return is 
\begin{itemize}
\item $R_{G,t} = \frac{356 - 345}{345} = 3.19\%$
\item $R_{A,t} = \frac{670-678}{678} = -1.12\%$
\end{itemize}
\begin{align*}
R_{p,t} & = (0.5044 \times 0.0319) + (0.4945 \times -0.0120)\\
& = 1.01\%
\end{align*}
You check!
%two ways to do that:  return of end portfolio value vs beginning value or the end value of return against beginning value. 
\end{frame}

%\begin{frame}{Variance of Portfolio}
%add this in later.  This will be the linear algebra example. 
%\end{frame}

\begin{frame}{Natural Logs and Exponentials}
Some notes on natural logs and exponentials
\begin{itemize}
\item $ln(0) = -\infty, ln(1) = 0$
\item $e^{-\infty} = 0, e^0 = 1, e^1 = 2.7183$
\item $\frac{dln(x)}{dx} = \frac{1}{x}, \frac{de^x}{dx}=e^x$
\item $ ln(e^x) = x, e^{ln(x)} = x$
\item $ln(x . y) = ln(x) + ln(y); ln \left ( \frac{x}{y} \right )= ln(x) - ln(y)$
\item $ln(x^y) = yln(x)$
\item $e^xe^y = e^{x+y}, e^xe^{-y} = e^{x-y}$
\item $(e^x)^y = e^{xy}$
\end{itemize}
\end{frame}

\begin{frame}{Continuously Compounded Return}
From Equation \ref{eqref:cc} and Equation \ref{eqref:ccln} 
\begin{align*}
1 + R_t &= e^{r_t}\\
R_t &= e^{r_t}-1\\
ln(1+R_t) &= r_t
\end{align*}
and, 
\begin{align*}
r_t = ln(1 + R_t) &= ln \left(\frac{P_t}{P_{p-t}}\right)\\
&= ln(P_t) - ln(P_{t-1})\\
& = p_t - p_{t-1}, \hspace{8pt} \text{where} \hspace{8pt} p_t = ln(P_t)  
\end{align*}

$r_t$ is always lower than $R_t$, why?
\end{frame}

\begin{frame}{Continuously compounded equations}
The equivalent of Equations \ref{eqref:pv}, \ref{eqref:cr} and \ref{eqref:ih}
\pause
\begin{block}{Present value}
\begin{equation}
V = e^{-Rn}FV
\end{equation}
\end{block}
\pause
\begin{block}{Compound return}
\begin{equation}
R = \frac{1}{n}ln \left (\frac{FV_n}{V} \right )
\end{equation}
\end{block}
\pause
\begin{block}{Investment horizon}
\begin{equation}
n = \frac{1}{R} ln \left ( \frac{FV_n}{V} \right )
\end{equation}
\end{block}
\end{frame}

\begin{frame}{Simple Returns and Compound Returns}
Intuition behind comparison
\begin{align*}
e^{r_t}&=e^{ln(1+R_t)}\\ 
&=e^{ln(P_t/P_{t-1})} \\
&= \frac{P_t}{P_{t-1}}\\
P_t &= P_{t-1}.e^{r_t}\\
\end{align*}
$P_{t-1}$ growing at $e^{r_t}$ gets to $P_t$
\end{frame}

\begin{frame}{Simple and Compound Returns}
\frametitle{Simple and Compound Returns}
\begin{center}
\includegraphics[height = 3.2in]{simpleandlog}
\end{center}
\end{frame} 

\begin{frame}{Multi Period Continuously Compounded Returns}
\begin{align*}
r_t(2) &= ln(1 + R_t(2))\\
&= ln \left ( \frac{P_t}{P_{t-2}} \right ) \\
&=p_t - p_{t-2}\\
\text{So that}, e^{rt(2)} = e^{ln(P_t/P_{t-2})}\\
& \rightarrow P_t = P_{t-2}e^{r(2)} 
\end{align*}
\end{frame}

\begin{frame}{Continuously Compounded Returns are Additive} 
\begin{align*}
r_t &= ln (1 + R_t) = ln \left ( \frac{P_t}{P_{t-1}} \right)\\
r_t(2) & = ln \left (\frac{P_t}{P_{t-1}}.\frac{P_{t-1}}{P_{t-2}} \right ) \\
& = ln \left ( \frac{P_t}{P_{t-1}} \right ) +  ln \left ( \frac{P_{t-1}}{P_{t-2}} \right ) \\
& = r_t + r_{t-1}
\end{align*}  
where $r_t = cc$, continuously compounded returns between months t-1 and t. 
\end{frame}


%\begin{frame}{Portfolios and continuously Compounded Returns}
%\begin{align*}
%R_{p,t} & = \sum_{i = 1}^n x_iR_{i,t}\\
%r_{p,t} & = ln(1 + R_{p,t})\\
%& =ln(1 + \sum_{i = 1}^n x_iR_{i,t}) \\
%& \neq \sum_{i = 1}^n x_ir_{i,t}
%\end{align*}
%It is approximately correct when $R_{p,t}$ is not large. 
%\end{frame}

\begin{frame}
Adapted from Modelling Financial Time Series by Eric Zivot
\end{frame}


\end{document}
