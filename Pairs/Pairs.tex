\documentclass[14pt,xcolor=pdftex,dvipsnames,table]{beamer}\usepackage[]{graphicx}\usepackage[]{color}
%% maxwidth is the original width if it is less than linewidth
%% otherwise use linewidth (to make sure the graphics do not exceed the margin)
\makeatletter
\def\maxwidth{ %
  \ifdim\Gin@nat@width>\linewidth
    \linewidth
  \else
    \Gin@nat@width
  \fi
}
\makeatother

\definecolor{fgcolor}{rgb}{0.345, 0.345, 0.345}
\newcommand{\hlnum}[1]{\textcolor[rgb]{0.686,0.059,0.569}{#1}}%
\newcommand{\hlstr}[1]{\textcolor[rgb]{0.192,0.494,0.8}{#1}}%
\newcommand{\hlcom}[1]{\textcolor[rgb]{0.678,0.584,0.686}{\textit{#1}}}%
\newcommand{\hlopt}[1]{\textcolor[rgb]{0,0,0}{#1}}%
\newcommand{\hlstd}[1]{\textcolor[rgb]{0.345,0.345,0.345}{#1}}%
\newcommand{\hlkwa}[1]{\textcolor[rgb]{0.161,0.373,0.58}{\textbf{#1}}}%
\newcommand{\hlkwb}[1]{\textcolor[rgb]{0.69,0.353,0.396}{#1}}%
\newcommand{\hlkwc}[1]{\textcolor[rgb]{0.333,0.667,0.333}{#1}}%
\newcommand{\hlkwd}[1]{\textcolor[rgb]{0.737,0.353,0.396}{\textbf{#1}}}%

\usepackage{framed}
\makeatletter
\newenvironment{kframe}{%
 \def\at@end@of@kframe{}%
 \ifinner\ifhmode%
  \def\at@end@of@kframe{\end{minipage}}%
  \begin{minipage}{\columnwidth}%
 \fi\fi%
 \def\FrameCommand##1{\hskip\@totalleftmargin \hskip-\fboxsep
 \colorbox{shadecolor}{##1}\hskip-\fboxsep
     % There is no \\@totalrightmargin, so:
     \hskip-\linewidth \hskip-\@totalleftmargin \hskip\columnwidth}%
 \MakeFramed {\advance\hsize-\width
   \@totalleftmargin\z@ \linewidth\hsize
   \@setminipage}}%
 {\par\unskip\endMakeFramed%
 \at@end@of@kframe}
\makeatother

\definecolor{shadecolor}{rgb}{.97, .97, .97}
\definecolor{messagecolor}{rgb}{0, 0, 0}
\definecolor{warningcolor}{rgb}{1, 0, 1}
\definecolor{errorcolor}{rgb}{1, 0, 0}
\newenvironment{knitrout}{}{} % an empty environment to be redefined in TeX

\usepackage{alltt}

% Specify theme
\usetheme{Madrid}
% See deic.uab.es/~iblanes/beamer_gallery/index_by_theme.html for other themes
\usepackage{caption}
\usepackage[comma, sort&compress]{natbib}
\usepackage{graphicx}
\usepackage{amsmath}
\bibliographystyle{agsm}
% Specify base color
\usecolortheme[named=OliveGreen]{structure}
% See http://goo.gl/p0Phn for other colors

% Specify other colors and options as required
\setbeamercolor{alerted text}{fg=Maroon}
\setbeamertemplate{items}[square]

\AtBeginSection[]{
  \begin{frame}
  \vfill
  \centering
  \begin{beamercolorbox}[sep=8pt,center,shadow=true,rounded=true]{title}
    \usebeamerfont{title}\insertsectionhead\par%
  \end{beamercolorbox}
  \vfill
  \end{frame}
}
% Title and author information
\title{Pairs Trading - Relative Value}
\author{Rob Hayward}
\IfFileExists{upquote.sty}{\usepackage{upquote}}{}
\begin{document}

\begin{frame}
\titlepage
\end{frame}

\begin{frame}{Outline}
\tableofcontents
\end{frame}

\section{Introduction}
\begin{frame}{Introduction}
This is a return full circle to the original hedge funds
\pause
\begin{itemize}[<+-| alert@+>]
\item Pairs trading will hope to provide an \emph{absolute return}
\item Paris trading can remove market risk and leave exposure to specific risk
\item Depends upon being able to short securities
\begin{itemize}
\item Professional ability to borrow securities
\item Use futures or contract-for-difference
\end{itemize}
\end{itemize}
\end{frame}

\section{Pairs Trading}
\begin{frame}{Pairs Trading}
The trade depends on a temporary breakdown in the correlation between two similar securities
\pause
\begin{itemize}[<+-| alert@+>]
\item LTCM strategy
\item When the usual relationship breaks down, speculate on a return to normality
\item Examples
\begin{itemize}
\item Tesco and Sainsbury
\item Coke and Pepsi
\item 5-year bond and 10-year bond
\item Gold and silver
\item Oil and products
\end{itemize}
\end{itemize}
\end{frame}

\begin{frame}{Pairs}
\begin{knitrout}
\definecolor{shadecolor}{rgb}{0.969, 0.969, 0.969}\color{fgcolor}
\includegraphics[width=\maxwidth]{figure/Coke-1} 

\end{knitrout}
\end{frame}

\begin{frame}{Coke vs Pepsi}
\begin{knitrout}
\definecolor{shadecolor}{rgb}{0.969, 0.969, 0.969}\color{fgcolor}
\includegraphics[width=\maxwidth]{figure/Coke2-1} 

\end{knitrout}
\end{frame}

\begin{frame}{Coke vs Pepsi}
\begin{knitrout}
\definecolor{shadecolor}{rgb}{0.969, 0.969, 0.969}\color{fgcolor}
\includegraphics[width=\maxwidth]{figure/CvP-1} 

\end{knitrout}
\end{frame}

\section{Co-integration}
\begin{frame}{Cointegration}
One method to look at the relationship would be the assess whether the two are cointegrated.
\pause
\begin{itemize}[<+-| alert@+>]
\item Can use the Engle-Granger method and check that the residuals from the regression are \emph{stationary}
\item If the residuals are stationary, we expect the previous relationship to be restored
\item An \emph{Error-Correction Model} can be used to assess the speed of the return to the set relationship
\end{itemize}
\end{frame}

\begin{frame}{Engle-Granger ECM}
\begin{block}{}
\begin{equation*}
\Delta y_t = \Phi_0 + \sum_{j=1} \Phi_j \Delta y_{t-j} + \sum_{h=1} \Phi_h \Delta x_{t-h} + \alpha \hat{u}_{t-1} + \varepsilon_t
\end{equation*}
\end{block}
Where, $y_t$ is value of the first pair at time $t$; $x_t$ is the other pair at time $t$; $\Phi_j, j = 0, 1, 2 \dots$, $\Phi_h, j = 0, 1, 2 \dots$  and $\alpha$ are parameters of to be estimated; $\hat{u_t}$ are the residuals from the regression $y_t = \beta_0 + \beta_1 x_t + \varepsilon_t$
\end{frame}
  
%\begin{frame}{OLS Coke and Pepsi}
%<<table>>=
%xtable(summary(eq))
%@
%\end{frame}
% This does not work.  
\begin{frame}{Regression: Coke and Pepsi}
% latex table generated in R 3.0.2 by xtable 1.7-1 package
% Fri Mar 21 05:51:48 2014
\begin{table}[ht]
\rowcolors{1}{OliveGreen!20}{OliveGreen!5}
\centering
\begin{tabular}{rrrrr}
  \hline
 & Estimate & Std. Error & t value & Pr($>$$|$t$|$) \\ 
  \hline
(Intercept) & 20.58 & 0.68 & 30.16 & 0.00 \\ 
  Coke & 1.37 & 0.02 & 56.18 & 0.00 \\ 
   \hline
\end{tabular}
\end{table}

The regression of Pepsi on Cole is only valid if the two are cointegrated.  Residuals must be checked. 

\end{frame}

\begin{frame}{Residuals}
\begin{knitrout}
\definecolor{shadecolor}{rgb}{0.969, 0.969, 0.969}\color{fgcolor}
\includegraphics[width=\maxwidth]{figure/Resid-1} 

\end{knitrout}
\end{frame}

%<<DF, echo=FALSE, message=FALSE>>=
%require(urca)
%da4 <- ur.df(eq$residuals, type = 'trend', lags = 3) 
%# Does not print anything.  the table is manual
%@
\begin{frame}{Dickey-Fuller}
\begin{table}[ht]
\rowcolors{1}{OliveGreen!20}{OliveGreen!5}
\centering
\begin{tabular}{rrrr}
  \hline
   & Test & 1pct& 5pct  \\ 
  \hline
$\tau$ & -2.52 & -3.96 & -3.41\\
 $\phi_2$ & 2.80 & 6.09 & 4.68\\ 
$\phi_3$ & 4.18 & 8.27 & 5.34\\ 
   \hline
\end{tabular}
\end{table}
Dickey-Fuller tests  show that the null of a unit root cannot be rejected ($\tau$).  Coke and Pepsi are not cointegrated. 
\end{frame}

\begin{frame}{The yield curve}
\begin{knitrout}
\definecolor{shadecolor}{rgb}{0.969, 0.969, 0.969}\color{fgcolor}
\includegraphics[width=\maxwidth]{figure/Yield-1} 

\end{knitrout}
 \end{frame}
 
\begin{frame}{Regression 2 on 10}
Dependent variable is 10 year
% latex table generated in R 3.2.0 by xtable 1.7-4 package
% Thu Feb 18 12:54:18 2016
\begin{table}[ht]
\rowcolors{1}{OliveGreen!20}{OliveGreen!5}
\centering
\begin{tabular}{rrrrr}
  \hline
 & Estimate & Std. Error & t value & Pr($>$$|$t$|$) \\ 
  \hline
(Intercept) & -0.20 & 0.01 & -13.21 & 0.00 \\ 
  `2-year` & 0.25 & 0.01 & 43.19 & 0.00 \\ 
   \hline
\end{tabular}
\end{table}

The explanatory variable is 2-year.
\end{frame}

\begin{frame}{Residuals}
\begin{knitrout}
\definecolor{shadecolor}{rgb}{0.969, 0.969, 0.969}\color{fgcolor}
\includegraphics[width=\maxwidth]{figure/resid-1} 

\end{knitrout}
\end{frame}

\begin{frame}{Testing unit root in residuals}
\begin{table}[ht]
\rowcolors{1}{OliveGreen!20}{OliveGreen!5}
\centering
\begin{tabular}{rrrr}
  \hline
   & Test & 1pct& 5pct  \\ 
  \hline
$\tau$ & -2.00 & -2.58 & -1.95\\
 $\phi_2$ & 1.99 & 6.43 & 4.59\\ 
$\phi_3$ & 3.63 & 8.27 & 5.34\\ 
   \hline
\end{tabular}
\end{table}
Dickey-Fuller tests  show that the null of a unit root is rejected ($\tau$) but that the zero restriction on the drift ($\phi_2$) and trend ($\phi_3$) cannot be rejected.   
\end{frame}

\begin{frame}{Error-Correction Model}
The error-correction model
\begin{block}{}
\begin{equation*}
\Delta x_t = \sum_{i = 1}^p \beta_{1,i} \Delta x_{t-i} + \sum_{i = 1}^p \beta_{2, i} \Delta y_{t-i} + \gamma ERR_{t-1} +\varepsilon_t
\end{equation*}
\end{block}
Where ERR are the residuals from the previous regression and $\beta_1$, $\beta_2$ and $\gamma$ are the parameters to be estimated.  A trend or drift term can be added.
\end{frame}

\begin{frame}{Error Correction Model}
dependent variable is $\Delta 10Y$, $R^2  = 0.5988$
\begin{table}[ht]
\rowcolors{1}{OliveGreen!20}{OliveGreen!5}
\centering
\begin{tabular}{lrrrr}
  \hline
Variable & Coefficient  & Std. Error & T-stat& Pr($>$$|$t$|$)  \\ 
  \hline
Intercept & 0.0018 & 0.0017 & 1.04 & 0.2974 \\ 
$\Delta 10Y_{t-1}$ & -0.1446 & 0.0357 & -4.05 & 0.0001 \\ 
$\Delta 2Y_{t-1}$ & 0.0826 & 0.0934 & 0.88 & 0.3764 \\ 
$\varepsilon_{t-1}$ & -0.0238 & 0.0140 & -1.70 & 0.0903 \\ 
   \hline
\end{tabular}
\end{table}
Can this be put into a trading model?
\end{frame}

\begin{frame}{Oil and gas}
\begin{knitrout}
\definecolor{shadecolor}{rgb}{0.969, 0.969, 0.969}\color{fgcolor}
\includegraphics[width=\maxwidth]{figure/OilGas-1} 

\end{knitrout}
\end{frame}

\begin{frame}{Brent-Gas ratio}
\begin{knitrout}
\definecolor{shadecolor}{rgb}{0.969, 0.969, 0.969}\color{fgcolor}
\includegraphics[width=\maxwidth]{figure/OilGasR-1} 

\end{knitrout}
\end{frame}



\begin{frame}{Brent-WTI}
\begin{knitrout}
\definecolor{shadecolor}{rgb}{0.969, 0.969, 0.969}\color{fgcolor}
\includegraphics[width=\maxwidth]{figure/bwt-1} 

\end{knitrout}
\end{frame}

\begin{frame}{Brent-WTI 2}
\begin{knitrout}
\definecolor{shadecolor}{rgb}{0.969, 0.969, 0.969}\color{fgcolor}
\includegraphics[width=\maxwidth]{figure/bwt2-1} 

\end{knitrout}
\end{frame}

\section{Relative value}
\begin{frame}{Relative value}
Now the focus is on one part of the pair out-performing
This can be based on, for example
\pause
\begin{itemize}[<+-| alert@+>]
\item Relative PE ratios
\item An assessment of competitive advantage
\item Products and brands
\item Country risk
\item Yield curve, duration and changes in short-term interest rates
\end{itemize}
\end{frame}

\end{document}

\begin{frame}{Yield curve predicts economy}
There is some research that suggests that the yield curve predicts the immediate economic outlook (one year ahead). 
\pause
\begin{itemize}[<+-| alert@+>]
\item Upward sloping yield curve indicates strong economic performance expected
\item Downward sloping yield curve indicates weak economic performance
\end{itemize}
\pause
Chinn, M. \& K. Kuko (2015), \emph{The predictive power of the yield curve across countries and time}
\end{frame}

